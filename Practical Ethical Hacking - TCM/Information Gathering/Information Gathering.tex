\documentclass[11pt,a4paper]{article}
\usepackage[utf8]{inputenc}
\usepackage[spanish]{babel}
\usepackage{amsmath}
\usepackage{amsfonts}
\usepackage{amssymb}
\usepackage{graphicx}
\usepackage[left=2cm,right=2cm,top=2cm,bottom=2cm]{geometry}
\author{Iker M. Canut}
\begin{document}
\title{Information Gathering}
\maketitle
\newpage

\section{hunter.io}
Hunter is a Domain Search. It gives a list of people that works in the organization: You get the first and last name and the most common pattern as far as email addresses are concerned. You can export all this information in a .csv. Maybe it tells you the department: Human Resources, IT/Engineering, Management, Executive, Legal, Sales, Support,... \\

This service may not list all the workers, but if you know that the email pattern is, for example, \textbf{\{f\}\{last\}@tesla.com}, and you know because of Linkedin, that "\textit{Iker Canut"} works there, you can probably assume his email is \textit{icanut@tesla.com}. This is crucial when we perform attacks (e.g password spraying in a login form).

\section{Breach Parse}
In hmaverickadams' Github, we can find a breach-parse. It's quite heavy, but it has emails and password from breaches: credentials got dumped out and you can use them. The bash script is for searching more easily. To illustrate this, you can write: \textbf{./breach-parse.sh @tesla.com tesla.txt}. Then, the results are extracted to three files: A master, passwords and users. You can take advantage if people utilize their work credentials and they log into websites.
\end{document}